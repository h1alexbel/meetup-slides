% MIT License
%
% Copyright (c) 2023 Aliaksei Bialiauski
%
% Permission is hereby granted, free of charge, to any person obtaining a copy
% of this software and associated documentation files (the "Software"), to deal
% in the Software without restriction, including without limitation the rights
% to use, copy, modify, merge, publish, distribute, sublicense, and/or sell
% copies of the Software, and to permit persons to whom the Software is
% furnished to do so, subject to the following conditions:
%
% The above copyright notice and this permission notice shall be included in all
% copies or substantial portions of the Software.
%
% THE SOFTWARE IS PROVIDED "AS IS", WITHOUT WARRANTY OF ANY KIND, EXPRESS OR
% IMPLIED, INCLUDING BUT NOT LIMITED TO THE WARRANTIES OF MERCHANTABILITY,
% FITNESS FOR A PARTICULAR PURPOSE AND NONINFRINGEMENT. IN NO EVENT SHALL THE
% AUTHORS OR COPYRIGHT HOLDERS BE LIABLE FOR ANY CLAIM, DAMAGES OR OTHER
% LIABILITY, WHETHER IN AN ACTION OF CONTRACT, TORT OR OTHERWISE, ARISING FROM,
% OUT OF OR IN CONNECTION WITH THE SOFTWARE OR THE USE OR OTHER DEALINGS IN THE
% SOFTWARE.

\documentclass{article}
\usepackage{..//cover}
\usepackage{..//slides}
\usepackage{..//inno}
\usepackage[normalem]{ulem}
\newcommand*\thetitle{Деловая}
\newcommand*\thesubtitle{и Организационная культура}
\begin{document}

  \plush{\defaultInnoTitlePage \innoDisclaimer}

  \plush[1]{%
    \innoSection{Деловая культура}
    \small
    Деловая культура описывает общие ценности и нормы, которые присущи организации и определяют ее подход к бизнесу, клиентам, сотрудникам и другим стейкхолдерам.
    Организационная культура относится к способу, которым работники взаимодействуют друг с другом внутри организации.
  }

  \plush[1]{%
    \innoSection{Элементы деловой культуры}
    \begin{itemize}
      \item ценности
      \item миссия
      \item видение
    \end{itemize}
  }

  \plush[1]{%
    \innoBanner{Личная мотивация как элемент деловой культуры}
    Каждый человек имеет свои личные мотивации, которые могут быть различными в зависимости от его характера, жизненного опыта и целей.
    Одной из ключевых задач менеджера является мотивация сотрудников к достижению общих целей организации.
  }

  \plush[1]{%
    \innoSection{Организационная культура}
    \small
    Организационная культура включает в себя способ, которым работники взаимодействуют друг с другом. Она может быть определена через взаимодействие, коммуникацию и стиль руководства.
    Хорошая организационная культура способствует командной работе, высокой производительности, мотивации сотрудников и привлечению талантов.
  }

  \plush[1]{%
    \innoSection{Коммуникации в Организационной культуры}
    \begin{itemize}
      \item Email
      \item Телефония
      \item IRC (Internet Relay Chat): Viber, Telegram, и т.д.
      \item многое другое\ldots
    \end{itemize}
  }

  \plush{
    \small
    \innoBanner{Взаимодействие Деловой и Организационной культуры}
    Деловая и организационная культуры организации тесно связаны друг с другом.
    Ценности, миссия и видение организации могут повлиять на организационную культуру и взаимодействия между сотрудниками.
    В свою очередь, организационная культура может помочь воплотить ценности и миссию в реальность.
  }

  \plush{
  \LARGE Спасибо за внимание! \\
  \LARGE Буду рад ответить на вопросы \\
  \small \center avbialiauski@gmail.com
  }
\end{document}
